\subsection*{I. Introduction}

These are the internal {\bfseries Python} guidelines for the \href{https://wp.nyu.edu/machinesinmotion/}{\tt machines-\/in-\/motion} group. The same guidelines are used in the \href{https://open-dynamic-robot-initiative.github.io/}{\tt Open Dynamic Robot Initiative}

The following rules present basic guidelines for our {\bfseries Python} code.

These guidelines may evolve in time so it is first good practice to check them upon creation of a new package or code refactoring.

\subsection*{II. Folder Structure and File Naming}


\begin{DoxyItemize}
\item Only {\itshape one} {\bfseries Python} package per git repository, and it must be located in {\ttfamily python/$<$catkin\+\_\+package\+\_\+name$>$/}.
\item Executable scripts should be placed in the {\ttfamily scripts/} folder. And should have a C\+Make {\bfseries install rule} that makes them executable upon installation.
\end{DoxyItemize}

\subsection*{II. Python Coding Guidelines}


\begin{DoxyItemize}
\item Regarding the style, follow \href{https://www.python.org/dev/peps/pep-0008/}{\tt P\+EP 8}.
\item Use \href{https://flake8.pycqa.org}{\tt flake8} and fix all issues it shows. This will ensure compliance with P\+EP 8 and also point out some critical issues like usage of undefined variables. It is recommended to install a plugin to automatically run flake8 in your favourite editor. See \mbox{[}List of flake8 Plugins for Popular Editors\mbox{]}(flake8\+\_\+plugins).
\item To automatically format you code, you may use \href{https://black.readthedocs.io}{\tt black}. When doing this, add {\ttfamily -\/-\/line-\/length 79} to not violate P\+EP 8. Note, however, that black requires Pytyon 3.\+6, which may not be available for everyone. Therefore its usage is not mandatory.
\end{DoxyItemize}

\subsection*{I\+II. Python Version}

{\bfseries Use Python 3.}

In case it is not possible to avoid {\bfseries Python 2} for some reason, add the following import at the top of your files to make it easier to port the code to {\bfseries Python 3} in the future and to avoid confusion about unexpected integer division\+: \begin{DoxyVerb}# Python 3 compatibility. It has to be the first import.
from __future__ import print_function, division\end{DoxyVerb}
 