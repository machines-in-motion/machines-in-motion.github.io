In dynamic graph manager, the sensor and control signals from and to the device are automatically exposed via rostopics.

Using rqt\+\_\+plot, it is possible to get a live view of the signals\textquotesingle{} values. In addition, it is possible to log values to file (using a tracer).\hypertarget{subpage_plot_plot_sec_tracer}{}\section{5.\+1/ Start and stop logging a trace}\label{subpage_plot_plot_sec_tracer}
You can start and stop logging the current registered tracer signals using\+: 
\begin{DoxyCode}
\textcolor{preprocessor}{# Signals from the device are logged by default. You can add your own signals as shown below.}
\hyperlink{namespacerobot}{robot}.start\_tracer()
\textcolor{preprocessor}{# Run your experiments.}
\textcolor{preprocessor}{robot.stop\_tracer()}
\end{DoxyCode}
 The log files are stored under \char`\"{}$\sim$/dynamic\+\_\+graph\+\_\+manager/$<$date$>$\+\_\+$<$time$>$/.\char`\"{}\hypertarget{subpage_plot_plot_sec_add_trace}{}\section{5.\+2/ Adding new signals to the tracer (for logging) and ros}\label{subpage_plot_plot_sec_add_trace}
By default, the input and output signals of the device are logged by the tracer and are exposed to ros. If you want to add other signals, you can do this using 
\begin{DoxyCode}
\hyperlink{namespacerobot}{robot}.add\_trace(entityName, sigName)
\hyperlink{namespacerobot}{robot}.add\_to\_ros(entityName, sigName)
\end{DoxyCode}


Expose the entity\textquotesingle{}s signal to ros and the tracer together. 
\begin{DoxyCode}
\hyperlink{namespacerobot}{robot}.add\_ros\_and\_trace(entityName, sigName)
\end{DoxyCode}
\hypertarget{subpage_plot_plot_sec_add_rai}{}\section{5.\+3/ Opening dynamic graph traces in R\+AI}\label{subpage_plot_plot_sec_add_rai}
To use the R\+AI plotting tool (\href{https://gitlab.tuebingen.mpg.de/SoW/amd-robot-plotting-framework}{\tt https\+://gitlab.\+tuebingen.\+mpg.\+de/\+So\+W/amd-\/robot-\/plotting-\/framework}), use this script (\href{https://git-amd.tuebingen.mpg.de/amd-clmc/dg_tools/blob/master/scripts/convert_dg_to_rai.py}{\tt https\+://git-\/amd.\+tuebingen.\+mpg.\+de/amd-\/clmc/dg\+\_\+tools/blob/master/scripts/convert\+\_\+dg\+\_\+to\+\_\+rai.\+py}) to convert a folder with dynamic graph dat files into a numpy compessed data file. R\+AI can read this compressed file. \begin{DoxyRefDesc}{Todo}
\item[\hyperlink{todo__todo000003}{Todo}]add a parser for the dynamic graph in R\+AI.\end{DoxyRefDesc}
\hypertarget{subpage_plot_plot_sec_add_rqt_plot}{}\section{5.\+4/ How to open rqt\+\_\+plot}\label{subpage_plot_plot_sec_add_rqt_plot}

\begin{DoxyItemize}
\item Make sure you have your dynamic graph manager started and the dynamic graph is running
\item Open the rqt\+\_\+plot by executing the following command in a terminal 
\begin{DoxyCode}
rosrun rqt\_plot rqt\_plot
\end{DoxyCode}

\item Add the values you want to plot using the input \char`\"{}\+Topic\char`\"{} at the top. Name the signal to plot like \char`\"{}/dg\+\_\+\+\_\+device\+\_\+\+\_\+ati\+\_\+ft\+\_\+sensor/\char`\"{} and then which vector entry you want to log \char`\"{}/dg\+\_\+\+\_\+device\+\_\+\+\_\+ati\+\_\+ft\+\_\+sensor/data\mbox{[}0\mbox{]}\char`\"{} to log the first entry from the \char`\"{}ati\+\_\+ft\+\_\+sensor\char`\"{} vector.
\end{DoxyItemize}\hypertarget{subpage_plot_plot_sec_add_rqt_troubleshooting}{}\section{5.\+5/ Troubleshooting with rqt in general\+:}\label{subpage_plot_plot_sec_add_rqt_troubleshooting}
If you ever have the case where the rqt plugins loads properly but the is not responding. You should delete the lock file that are located in \char`\"{}$\sim$/.\+config/ros.\+org\char`\"{}. 
\begin{DoxyCode}
rm -fr ~/.config/\hyperlink{namespaceros}{ros}.org\textcolor{comment}{/*}
\end{DoxyCode}
\hypertarget{subpage_plot_plot_sec_add_timing}{}\section{5.\+6/ Logging of motor and control process timing}\label{subpage_plot_plot_sec_add_timing}
The dynamic graph manager has a build in support for logging a set of timing informations. These include\+:

\hypertarget{subsubpage_stl_simplify_cad}{}
\tabulinesep=1mm
\begin{longtabu} spread 0pt [c]{*{2}{|X[-1]}|}
\caption{}\label{subsubpage_stl_simplify_cad}\\
\hline
dg\+\_\+timer.\+dat &How long it took to run the dynamic graph process in s. \\\cline{1-2}
dg\+\_\+active\+\_\+timer.\+dat &How long the dynamic-\/graph process was active in s. \\\cline{1-2}
dg\+\_\+sleep\+\_\+timer.\+dat &How long the dynamic-\/grpah process was sleeping in s. \\\cline{1-2}
hwc\+\_\+active\+\_\+timer.\+dat &How long the motor process was active in s. That is the time without sleeping. At the point of writing this includes\+:
\begin{DoxyItemize}
\item Reading the control signals from the control process or run the saftey controller
\item Execute the control signals on the hardware by calling set\+\_\+motor\+\_\+controls\+\_\+from\+\_\+map
\item Read the new sensor values from the robot by calling get\+\_\+sensors\+\_\+to\+\_\+map
\item Process the user commands 
\end{DoxyItemize}\\\cline{1-2}
hwc\+\_\+sleep\+\_\+timer.\+dat &How long the motor process was sleeping in s. The sleeping is done to ensure the motor process runs as the desired control frequency. \\\cline{1-2}
hwc\+\_\+timer.\+dat &How long the motor process was sleeping in s. The sleeping is done to ensure the motor process runs as the desired control frequency. \\\cline{1-2}
hwc\+\_\+timer.\+dat &The total time passed between two sleeping times of the motor process in s. \\\cline{1-2}
\end{longtabu}


By default these files are created in the dynamic graph manager log directory. However, the files are empty. This is as by default a length zero history is recorded. To record N timesteps, the parameter \char`\"{}debug\+\_\+timer\+\_\+history\+\_\+length\+: 10000\char`\"{} (where N=10000 here, which corresponds to 10 seconds at 1 k\+Hz) has to be specified at the top level of the robot yaml file. 